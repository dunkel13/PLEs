\documentclass{beamer}
\usepackage{beamerthemesplit}
\usepackage{latexsym}
\usepackage{eurosym}
\usepackage[activeacute,english]{babel}
\usepackage{ae,aecompl}
\usepackage{graphicx}
\usepackage{amsfonts}
\usepackage{tikz}
\usetikzlibrary{arrows,automata}
\usetikzlibrary{calc,3d}
\usepackage{rotating}
\usepackage{color}
\usepackage{moreverb}
\usepackage{longtable}
\usepackage{tabularx}
\usepackage{verbatim}
\usepackage{listings}
\usepackage{array}
\usepackage[refpages]{gloss}
\usepackage{amsmath}
\usepackage{amssymb}
\usepackage{multirow}
%\usepackage{bm, bbm}

\def \ind  {\stackrel{{\rm \tiny IND}}{\sim}}  % definir ind
\def \iid  {\stackrel{{\rm \tiny IID}}{\sim}}  % definir iid

\usetheme{Boadilla}
\title[Inferencia Basada en el Modelo]{Inferencia Basada en el Modelo para Dise\~nos de Muestreo Convencionales}
\author[Trujillo, Leonardo]{Leonardo Trujillo}
\institute[]{Departamento de Estadistica\\Universidad Nacional de Colombia\\Seminario del Departamento}
\date{17-05-2018}

\begin{document}

%1
\frame{\titlepage}

%2
\begin{frame}
\frametitle{Outline}
\tableofcontents
\end{frame}
%\frame{\tableofcontents}

%3
\section{Introduccion}
 \tableofcontents[currentsection]

%4
\begin{frame}[allowframebreaks*]
\frametitle{Inferencia Basada en el Dise\~no}
En inferencia basada en el dise\~no, los valores $y_1, ..., y_N$ son considerados como constantes fijas. La idea es inferir acerca de parametros descriptivos desconocidos en la poblacion finita $U$. Las observaciones disponibles se limitan a aquellas observadas en una muestra $s$ y la poblacion finita $U$ es el objetivo de la inferencia. La estructura estocastica necesaria es inducida por el dise\~no muestral donde la muestra $s$ es el elemento aleatorio. Se obtiene un estimador de la varianza del dise\~no. La inferencia se obtiene con respecto a la distribucion de estadisticas sobre muestras repetidas S generadas por el diseño muestral.
\end{frame}

%5
\begin{frame}[allowframebreaks*]
\frametitle{Inferencia Basada en el Modelo}
En inferencia basada en el modelo, se recurre a un modelo de superpoblacion pero la inferencia sigue siendo para la poblacion finita y sus parametros. La definicion en inferencia estadistica para una muestra aleatoria se refiere a un conjunto de variables aleatorias $Y_1, ..., Y_N$ i.i.d (con una distribucion normal con media $\mu$ y varianza $\sigma^2$. Podemos extender esta idea ahora a variables aleatorias $Y_1, ..., Y_N$ generadas a partir de cierto modelo $\xi$. Los valores reales para la poblacion finita $y_1, ..., y_N$ son una realizacion de las variables aleatorias. La distribucion de probabilidad conjunta de $Y_1, ..., Y_N$ proporciona este vinculo entre las unidades que estan en la muestra y las que no estan. Muestreamos ${y_k, k \in s}$ para predecir los valores no observados ${y_k, k \notin s}$.
\end{frame}

%6
\begin{frame}[allowframebreaks*]
\frametitle{Ejemplo}
Para una region y un periodo de tiempo dado se presentan datos de su poblacion de hospitales (Morales, 2015). Se quiere estimar $T=\sum\limits_{k=1}^{N}y_k$. Para ello se extrae una muestra $s$ de tama\~no $n$=31, excluyendo el caso 16. Observese que $\sum\limits_{k \in s}y_k$ y que $T=\sum\limits_{k \in s}y_k + y_{16}$. Por tanto estimar $T$ equivale a predecir $y_{16}$. Supongamos que ademas se verifica un modelo $M$ de regresion lineal simple, es decir $E_M[Y_k]=\beta x_k$ $k=1, ...,N$ con $Cov_M(Y_k,Y_l)=\sigma^2 x_k$ si $k=l$ y 0 en otro caso. $\beta$ es un parametro desconocido que se debe estimar.
\end{frame}

%6
\begin{frame}[allowframebreaks*]
\frametitle{Ejemplo}
\begin{tabular}{|c|c|c|}
                                                                                  \hline
                                                                                  % after \\: \hline or \cline{col1-col2} \cline{col3-col4} ...
                                                                                  Hospital & Camas(x) & Ingresos(y) \\
                                                                                  1 & 49 & 173 \\
                                                                                  2 & 50 & 260 \\
                                                                                  3 & 64 & 225 \\
                                                                                  4 & 70 & 209 \\
                                                                                  5 & 99 & 346 \\
                                                                                  6 & 100 & 383 \\
                                                                                  7 & 100 & 318 \\                                                                  8 & 100 & 373 \\
                                                                                  9 & 128 & 577 \\
                                                                                  10 & 184 & 481 \\
                                                                                  11 & 224 & 590 \\
                                                                                  12 & 227 & 732 \\
                                                                                  13 & 231 & 931 \\
                                                                                  14 & 233 & 684 \\
                                                                                  15 & 244 & 858 \\
                                                                                  \textbf{16} & \textbf{260} & \textbf{1076} \\
                                                                                  \hline
                                                                                \end{tabular}
\begin{tabular}{|c|c|c|}
                                                                                  \hline
                                                                                  % after \\: \hline or \cline{col1-col2} \cline{col3-col4} ...
                                                                                  Hospital & Camas(x) & Ingresos(y) \\
                                                                                  17 & 275 & 1201 \\
                                                                                  18 & 279 & 754 \\
                                                                                  19 & 303 & 715 \\
                                                                                  20 & 309 & 985 \\
                                                                                  21 & 347 & 1166 \\
                                                                                  22 & 350 & 1173 \\
                                                                                  23 & 373 & 787 \\                                                                 24 & 411 & 808 \\
                                                                                  25 & 417 & 1369 \\
                                                                                  26 & 451 & 1584 \\
                                                                                  27 & 523 & 1232 \\
                                                                                  28 & 549 & 1547 \\
                                                                                  29 & 551 & 1645 \\
                                                                                  30 & 558 & 1152 \\
                                                                                  31 & 562 & 2116 \\
                                                                                  32 & 591 & 999 \\
                                                                                  \hline
                                                                                \end{tabular}
\end{frame}

%6
\begin{frame}[allowframebreaks*]
\frametitle{Estimador BLUE}
El estimador lineal insesgado de minima varianza (blue - best linear unbiased estimator) se obtiene minimizando la suma ponderada de cuadrados de los errores
\begin{equation*}
  SPCE = \sum\limits_{k \in s} \frac{1}{\sigma^2 x_k} (y_k - \beta x_k)^2
\end{equation*}
es decir,
\begin{equation*}
  0=\frac{\partial SCPE}{\partial \beta}= -\sum\limits_{k \in s}\frac{2(y_k - \beta x_k) x_k}{\sigma^2 x_k}
\end{equation*}
Por lo tanto,
\begin{equation*}
\hat{\beta}=\frac{\sum\limits_{k \in s} y_k}{\sum\limits_{k \in s} x_k}=\frac{28641}{9938}=2,882
\end{equation*}
\end{frame}

%6
\begin{frame}[allowframebreaks*]
\frametitle{Estimador BLUE}
En un analisis de regresion convencional se construiria un intervalo de confianza para $\beta$ o se contrastaria $H_0: \beta$ = 0, pero aqui $\hat{\beta}$ es solo un paso intermedio para llegar al objetivo de estimar $T$. En el hospital 16, se tiene $x_{16}$=260, $\hat{y}_{16}=\hat{\beta} x_{16}$=(2,882)(260)=749 y por lo tanto
\begin{equation*}
  \hat{T}=\sum\limits_{k \in s}+\hat{y}_{16}=28641+749=29390
\end{equation*}
Observese que en este caso el error relativo es
\begin{equation*}
  \frac{|\hat{T}-T|}{T} 100\% =\frac{|29390-29717}{29717} 100\% \cong 1\%
\end{equation*}
que puede considerarse como moderado.
\end{frame}

%6
\begin{frame}[allowframebreaks*]
\frametitle{Estimador BLUE}
El ejemplo anterior se puede generalizar al caso
\begin{equation*}
  T=\sum\limits_{k \in s} y_k + \sum\limits_{k \in U-s} y_k
\end{equation*}
En este contexto el estimador de $T$ sera
\begin{align*}
  \hat{T}&=\sum\limits_{k \in s} y_k + \sum\limits_{k \in U-s} \hat{\beta} x_{k} = \sum\limits_{k \in s} y_k + \frac{\sum\limits_{k \in s} y_k}{\sum\limits_{k \in s} x_k} \sum\limits_{k \in U-s} x_k \\
  &= \left(\frac{\sum\limits_{k \in s} y_k}{\sum\limits_{k \in s} x_k}\right) \sum\limits_{k \in U} x_k = N\bar{y}_s \frac{\bar{x}_U}{\bar{x}_s}
\end{align*}
(Estimador de Razon)
\end{frame}

%6
\begin{frame}[allowframebreaks*]
\frametitle{Modelos y Dise\~nos de Muestreo Convencionales}
Consideraremos ahora varios dise\~nos de muestreo convencionales en la literatura de inferencia basada en el dise\~no y propondremos un modelo relacionado bajo la inferencia basada en el modelo con el fin de estimar un total poblacional de acuerdo a lo presentado en Lohr (2000, Sampling - Design and Analysis, Duxbury). Consideraremos los siguientes dise\~nos muestrales: muestreo aleatorio simple, muestreo estratificado, muestreo de conglomerados, muestreo con probabilidades proporcionales al tama\~no, estimador de razon, estimador de regresion). Igualmente consideramos al final el problema de la determinacion del tama\~no de muestra.
\end{frame}

%7
\section{Muestreo Aleatorio Simple}
 \tableofcontents[currentsection]

%8
\begin{frame}[allowframebreaks*]
\frametitle{Modelo para Muestreo Aleatorio Simple}
{\footnotesize
Un modelo para este dise\~no se plantea en Lohr (2000) como
$$\xi: y_k = \mu + U_k \hspace{1.5cm} k=1,\ldots,N$$
donde se tiene que
\begin{align*}
E_{\xi}(U_k)&=0 \\
E_{\xi}(U_k^2)&=\sigma^2\\
E_{\xi}(U_k U_l)&=0, \hspace{1cm} k,l = 1, 2, \ldots, N \hspace{.5cm} \text{con} \hspace{.5cm} k \neq l
\end{align*}

Se observa que se puede expresar el total de la siguiente forma:

\begin{equation*}
t=\sum\limits_{i=1}^Ny_k=\sum\limits_{S}y_k+\sum\limits_{S^c}y_k
\end{equation*}}
\end{frame}

%9
\begin{frame}[allowframebreaks*]
\frametitle{Modelo para Muestreo Aleatorio Simple}
La cual es una realizacion de la v. a.
\begin{align*}
T&=\sum\limits_{i=1}^NY_k=\sum\limits_{S}Y_k+\sum\limits_{S^c}Y_k\\
&\Rightarrow \hat T=\sum\limits_{S}Y_k+\sum\limits_{S^c}\hat Y_k
\end{align*}

Dado el modelo que se tomo, se puede observar:

\begin{align*}
\xi:Y_k&=\mu+U_k\\
E_{\xi}(Y_k)&=E_{\xi}(\mu+U_k)\\
&=\mu+E_{\xi}(U_k)\\
&=\mu+0=\mu\\
E_{\xi}(Y_k)&=\mu
\end{align*}

Como $$E_{\xi}(Y_k)=\mu\Rightarrow \hat {Y}_k=\hat\mu$$
\end{frame}

%10
\begin{frame}[allowframebreaks*]
\frametitle{Minimos Cuadrados Ordinarios}
{\footnotesize
Mediante la metodologia de minimos cuadrados ordinarios se tiene:

\begin{align*}
\sum\limits_{k \in S}(Y_k-E(Y_k))^2&=\sum_{k \in S}(Y_k-\hat \mu)^2\rightarrow SCE\\
\frac{\partial SCE}{\partial\mu}&=-2\sum\limits_{k \in S}(Y_k-\hat\mu)=0\\
&\Rightarrow \sum\limits_{k \in S}Y_k-\sum\limits_{k \in S}\hat\mu=0\\
&\Rightarrow \sum\limits_{k \in S}Y_k=n\hat\mu\\
&\Rightarrow \sum\limits_{k \in S}\frac{Y_k}{n}=\hat\mu\\
&\Rightarrow \hat\mu=\bar Y
\end{align*}

con $\bar Y= \sum\limits_{i=1}^n\frac{Y_i}{n}= \sum\limits_{k \in S}\frac{Y_k}{n}$
}
\end{frame}

%11
\begin{frame}[allowframebreaks*]
\frametitle{Estimacion del Total Poblacional}
{\footnotesize
Dado esto se tiene:
\begin{align*}
\hat T&=\sum_S Y_k+\sum_{S^c}\hat Y_k\\
&=\sum_S Y_k+\sum_{S^c}\hat \mu\\
&=\sum_S Y_k+\sum_{S^c}\left(\sum_S\frac{Y_k}{n}\right)\\
&=\sum_SY_k+\frac{(N-n)}{n}\sum_S Y_k\\
&=\sum_SY_k+\frac{N}{n}\sum_SY_k-\frac{n}{n}\sum_SY_k\\
&=\frac{N\sum_S Y_k}{n}=N\bar Y_S=N\hat\mu\\
&\Rightarrow \hat{\bar {Y}}_U=\frac{\hat T}{N}=\bar Y_S=\hat\mu
\end{align*}
$Y_S$ es insesgado para $\mu$? BLUP. $Y_S$ es insesgado para $\bar Y_U?$. $\hat T$ es insesgado para $N\mu$?. $\hat T$ es insesgado para $T$?
}
\end{frame}

%12
\begin{frame}[allowframebreaks*]
\frametitle{Insesgamiento}
Ahora se muestra que $\hat T$ es insesgado para $T$.

\begin{align*}
E[\hat T]&=E[\sum_S Y_k+\sum_{S^c}\hat Y_k]=E[\sum_S Y_k]+E[\sum_{S^c}\hat Y_k] \\
&=E[\sum_S Y_k]+E[\sum_{S^c}\hat \mu]=E[\sum_S Y_k]+E[\sum_{S^c}\bar Y_S]
\end{align*}

Dado que una v.a no puede ser el valor esperado de otra v.a, no se puede observar que sea insesgado por este metodo. Se va a observar que el promedio de $\hat T-T$ bajo varias realizaciones es 0.
\end{frame}

%13
\begin{frame}[allowframebreaks*]
\frametitle{Insesgamiento}
{\footnotesize
\begin{align*}
E_\xi[\hat T-T]&=E_\xi[\hat T]-E_\xi[T]\\
&=E_\xi\left[\frac{N}{n}\sum_SY_k\right]-E_\xi\left[\sum_UY_k\right]\\
&=\frac{N}{n}E_\xi\left[\sum_SY_k\right]-E_\xi\left[\sum_UY_k\right]\\
&=\frac{N}{n}\sum_SE_\xi\left[Y_k\right]-E_\xi\sum_U\left[Y_k\right]\\
&=\frac{N}{n}\sum_S\mu-\sum_U\mu\\
&=\frac{N}{n}n\mu-N\mu=N\mu-N\mu=0
\end{align*}
El estimador $\hat T$ es insesgado para $T$.

\begin{align*}
E_\xi[\bar Y_S-\bar Y_U]&=E_\xi[\bar Y_S]-E_\xi[\bar Y_U]\\
&=\sum_S\frac{E_\xi[Y_k]}{n}-\sum_U\frac{E_\xi[Y_k]}{N}=\frac{n}{n}\mu-\frac{N}{N}\mu=0
\end{align*}
}
\end{frame}

%14
\begin{frame}[allowframebreaks*]
\frametitle{Varianza}
\begin{align*}
V_{\xi}(\hat\mu)&=V_\xi(\bar Y_S)=V_\xi\left(\sum_S\frac{Y_k}{n}\right)\\
&\stackrel{ind}{=}\sum_S\frac{V_\xi(Y_k)}{n^2}=\sum_S\frac{\sigma^2}{n^2}=\frac{\sigma^2}{n}
\end{align*}

Entonces se tiene:

\begin{align*}
V_\xi(\hat T)&=V_\xi\left(\frac{N\sum_S Y_k}{n}\right)=\frac{N^2}{n^2}V_\xi(\sum_SY_k)\stackrel{\text{ind}}{=}\frac{N^2}{n^2}n\sigma^2=\frac{N^2}{n}\sigma^2\\
V_\xi(\hat {\bar {Y}}_U)&=V_\xi\left(\sum_S\frac{Y_k}{n}\right)=\frac{1}{n^2}V_\xi(\sum_S Y_k)\stackrel{\text{ind}}{=}\frac{1}{n^2}n\sigma^2=\frac{1}{n}\sigma^2
\end{align*}
\end{frame}

%15
\begin{frame}[allowframebreaks*]
\frametitle{Error Cuadratico Medio}
{\footnotesize
\begin{align*}
ECM[(\hat T-T)^2]&=E_\xi[(\hat T-T)]^2=E_\xi\left[\left(\frac{N}{n}\sum_SY_k-\sum_UY_k\right)^2\right]\\
&=E_\xi\left[\left(\frac{N}{n}\sum_SY_k-\sum_SY_k-\sum_{S^c}Y_k\right)^2\right]\\
&=E_\xi\left[\left(\left(\frac{N}{n}-1\right)\sum_SY_k-\sum_{S^c}Y_k\right)^2\right]\\
&=E_\xi\left[\left(\left(\frac{N}{n}-1\right)\sum_SY_k-\sum_{S^c}Y_k-\left(\frac{N}{n}-1\right)n\mu+\left(\frac{N}{n}-1\right)n\mu\right)^2\right]\\
&=E_\xi\left[\left(\left(\frac{N}{n}-1\right)\left(\sum_SY_k-n\mu\right)-\left(\sum_{S^c}Y_k-\left(\frac{N}{n}-1\right)n\mu\right)\right)^2\right]\\
&=E_\xi\left[\left(\frac{N}{n}-1\right)^2\left(\sum_SY_k-n\mu\right)^2+\left(\sum_{S^c}Y_k-\left(\frac{N}{n}-1\right)n\mu\right)^2\right. \\ &\left.-2\left(\frac{N}{n}-1\right)\left(\sum_SY_k-n\mu\right)\left(\sum_{S^c}Y_k-\left(\frac{N}{n}-1\right)n\mu\right)\right]
\end{align*}
}
\end{frame}

%16
\begin{frame}[allowframebreaks*]
\frametitle{Error Cuadratico Medio}
{\footnotesize
\begin{align*}
&=\left(\frac{N}{n}-1\right)^2
E_\xi\left[\left(\sum_SY_k-n\mu\right)^2\right]+E_\xi\left[\left(\sum_{S^c}Y_k-\left(\frac{N}{n}-1\right)n\mu\right)^2\right] \\ &-2\left(\frac{N}{n}-1\right)E_\xi\left[\left(\sum_SY_k-n\mu\right)\left(\sum_{S^c}Y_k-\left(\frac{N}{n}-1\right)n\mu\right)\right]\\
&=\left(\frac{N}{n}-1\right)^2E_\xi\left[n^2\left(\sum_S\frac{Y_k}{n}-\mu\right)^2\right]+E_\xi\left[\left(\sum_{S^c}Y_k-\left(N-n\right)\mu\right)^2\right] \\
&-2\left(\frac{N}{n}-1\right)cov\left(\sum_SY_k,\sum_{S^c}Y_k\right)\\
&\stackrel{\text{ind}}{=}\left(\frac{N}{n}-1\right)^2 n^2E_\xi\left[\left(\sum_S\frac{Y_k}{n}-\mu\right)^2\right]+\left(N-n\right)^2E_\xi\left[\left(\sum_{S^c}\frac{Y_k}{(N-n)}-\mu\right)^2\right]
\end{align*}
}
\end{frame}

%17
\begin{frame}[allowframebreaks*]
\frametitle{Error Cuadratico Medio}
{\footnotesize
\begin{align*}
ECM[(\hat T-T)^2]&=\left(\frac{N}{n}-1\right)^2n^2\frac{\sigma^2}{n}+\left(N-n\right)^2\frac{\sigma^2}{(N-n)}\\
&=\left(\frac{N}{n}-1\right)^2n\sigma^2+\left(N-n\right)\sigma^2\\
&=\left(\frac{N-n}{n}\right)^2n\sigma^2+\left(N-n\right)\sigma^2\\
&=\sigma^2\left[\frac{(N-n)^2}{n^2}n+\left(N-n\right)\right]\\
&=\sigma^2\left[\frac{N^2-2Nn+n^2}{n}+\left(N-n\right)\right]\\
&=\sigma^2\left[\frac{N^2-2Nn+n^2+Nn-n^2}{n}\right]\\
&=\sigma^2\left[\frac{N^2-Nn}{n}\right]=\frac{N^2}{n}\sigma^2\left[1-\frac{n}{N}\right]\\
&=N^2\frac{(1-f)}{n}\sigma^2
\end{align*}}
\end{frame}

%18
\section{Muestreo Estratificado}
 \tableofcontents[currentsection]

%19
\begin{frame}[allowframebreaks*]
\frametitle{Modelo para Muestreo MAS Estratificado}
{\footnotesize
Un modelo para el dise\~no MAS estratificado seria
$$\xi: Y_{hk}=\mu_h+U_{hk}\hspace{0.5cm}k=1,\ldots,N\hspace{0.5cm}h=1,\ldots,H$$
donde se tiene que:
\begin{align*}
E_{\xi}(U_{hk})&=0\\
E_{\xi}(U_{hk}^2)&=\sigma_h^2\\
E_{\xi}(U_{hk}U_{hl})&=0\\
E_{\xi}(Y_{hk})&=\mu_h\\
Var_{\xi}(Y_{hk})&=\sigma_h^2\\
Cov_{\xi}(Y_{hk}Y_{hl})&=0 \hspace{.5cm} k,l&=1,\ldots,N \hspace{.5cm} k&\neq l
\end{align*}
El total global se puede observar como:
\begin{equation*}
T=\sum\limits_{h=1}^HT_h
\end{equation*}
donde los $T_h$ son los totales de cada uno de los estratos.}
\end{frame}

%14
\begin{frame}[allowframebreaks*]
\frametitle{Notacion}
{\footnotesize
\begin{align*}
T_h&=\sum\limits_{k=1}^{N_h}Y_{hk}\\
T&=\sum\limits_{h=1}^HT_h=\sum\limits_{h=1}^H\sum\limits_{k=1}^{N_h}Y_{hk}=\sum\limits_{h=1}^H\left(\sum_SY_{hk}+\sum_{S^c}Y_{hk}\right)\\
\Rightarrow \hat T&=\sum\limits_{h=1}^H\left(\sum_SY_{hk}+\sum_{S^c}\hat Y_{hk}\right)\\
\end{align*}
pero se tiene que:
\begin{align*}
E(Y_{hk})&=E(\mu_h+U_{hk})=E(\mu_h)+E(U_{hk})=\mu_h+0=\mu_h\\
\Rightarrow \hat Y_{hk}&=\hat\mu_h
\end{align*}}
\end{frame}

%15
\begin{frame}[allowframebreaks*]
\frametitle{Minimos Cuadrados}
{\footnotesize
Por minimos cuadrados ordinarios se tiene:
\begin{align*}
\sum\limits_{k=1}^{n_h}\left(Y_{hk}-E (Y_{hk})\right)^2&=\sum\limits_{k=1}^{n_h}\left(Y_{hk}-\mu_{h}\right)^2\rightarrow \text{Suma de Cuadrados del Error } (SCE)\\
\frac{\partial SCE}{\partial\hat\mu}&=-2\sum\limits_{k=1}^{n_h}\left(Y_{hk}-\hat \mu_{h}\right)=0\\
&\Rightarrow\sum\limits_{k=1}^{n_h} Y_{hk}-\sum\limits_{k=1}^{n_h}\hat\mu_h=0\\
&\Rightarrow\sum\limits_{k=1}^{n_h}Y_{hk}=\sum\limits_{k=1}^{n_h}\hat\mu_h\\
&\Rightarrow\frac{1}{n_h}\sum\limits_{k=1}^{n_h}Y_{hk}=\hat\mu_h
\end{align*}}
\end{frame}

%16
\begin{frame}[allowframbreaks*]
\frametitle{Estimacion}
{\footnotesize
Entonces un estimador para el total seria:
\begin{align*}
\hat T_h&=\sum_SY_{hk}+\sum_{S^c}\hat Y_{hk}\\
&=\sum_S Y_{hk}+\sum_{S^c}\hat\mu_h\\
&=\sum_S Y_{hk}+\sum_{S^c}\left(\frac{1}{n_h}\sum\limits_{k=1}^{n_h}Y_{hk}\right)\\
&=\sum_S Y_{hk}+\frac{\left(N_h-n_h\right)}{n_h}\sum\limits_{k=1}^{n_h}Y_{hk}\\
&=\frac{N_h}{n_h}\sum\limits_{k=1}^{n_h}Y_{hk}+\sum_SY_{hk}-\sum\limits_{k=1}^{n_h}Y_{hk}\\
&=\frac{N_h}{n_h}\sum\limits_{k=1}^{n_h}Y_{hk}\\
&=\frac{N_h}{n_h}\sum_S Y_{hk}
\end{align*}}
\end{frame}

%17
\begin{frame}[allowframebreaks*]
\frametitle{Insesgamiento}
\begin{align*}
E_\xi[\hat T_h-T_h]&=E_\xi[\hat T_h]-E_\xi[T_h]\\
&=E_\xi\left[\frac{N_h}{n_h}\sum_SY_{hk}\right]-E_\xi\left[\sum\limits_{k=1}^{N_h}Y_{hk}\right]\\
&=\frac{N_h}{n_h}\sum_SE_\xi\left[Y_{hk}\right]-\sum\limits_{k=1}^{N_h}E_\xi\left[Y_{hk}\right]\\
&=\frac{N_h}{n_h}\sum_S\mu_h-\sum_{k=1}^{N_h}\mu_h\\
&=\frac{N_h}{n_h}n_h\mu_h-N_h\mu_h\\
&=N_h\mu_h-N_h\mu_h=0
\end{align*}
\end{frame}

%18
\begin{frame}[allowframebreaks*]
\frametitle{Error Cuadratico Medio}
{\footnotesize
\begin{align*}
\centering
ECM_\xi&=E_\xi\left[(\hat T_h-T_h)^2\right]\\
&=E_\xi\left[\left(\frac{N_h}{n_h}\sum_SY_{hk}-\sum\limits_{k=1}^{N_h}Y_{hk}\right)^2\right]\\
&=E_\xi\left[\left(\frac{N_h}{n_h}\sum_SY_{hk}-\sum_{S}Y_{hk}-\sum_{S^c}Y_{hk}\right)^2\right]\\
&=E_\xi\left[\left(\left(\frac{N_h}{n_h}-1\right)\sum_SY_{hk}-\sum_{S^c}Y_{hk}\right)^2\right]\\
&=E_\xi\left[\left(\left(\frac{N_h}{n_h}-1\right)\sum_SY_{hk}-\sum_{S^c}Y_{hk}-\left(\frac{N_h}{n_h}-1\right)n_h\mu_h+\left(\frac{N_h}{n_h}-1\right)n_h\mu_h\right)^2\right]\\
&=E_\xi\left[\left(\left(\frac{N_h}{n_h}-1\right)\left(\sum_SY_{hk}-n_h\mu_h\right)-\left(\sum_{S^c}Y_{hk}-\left(\frac{N_h}{n_h}-1\right)n_h\mu_h\right)\right)^2\right]\\
\end{align*}}
\end{frame}

%19
\begin{frame}[allowframebreaks*]
\frametitle{Error Cuadratico Medio}
{\footnotesize
\begin{align*}
&=\sigma^2_h\left(\left(\frac{N_h}{n_h}-1\right)^2n_h+(N_h-n_h)\right)\\
&=\sigma^2_h\left(\left(\frac{N_h-n_h}{n_h}\right)^2n_h+(N_h-n_h)\right)\\
&=\sigma^2_h\left(\left(\frac{N^2_h+n_h^2-2n_hN_h}{n_h^2}\right)n_h+N_h-n_h\right)\\
&=\sigma^2_h\left(\frac{N^2_h+n_h^2-2n_hN_h+N_hn_h-n_h^2}{n_h}\right)\\
&=\sigma^2_h\left(\frac{N^2_h-n_hN_h}{n_h}\right)\\
&=\frac{\sigma^2_h}{n_h}N_h^2\left(1-\frac{n_h}{N_h}\right)\\
&=N_h^2\frac{\sigma^2_h}{n_h}(1-f)
\end{align*}}
\end{frame}

%20
\begin{frame}[allowframebreaks*]
\frametitle{Error Cuadratico Medio}
{\footnotesize
\begin{equation*}
ECM_\xi(T)=E_\xi[(\hat T-T)^2]=E_\xi\left[\left(\sum\limits_{h=1}^H(\hat T_h-T_h)\right)^2\right]
\end{equation*}
Se puede observar que se tiene:
\begin{align*}
E_\xi\left[\left(\sum\limits_{h=1}^H(\hat T_h-T_h)\right)^2\right]&=E_\xi\left[\sum\limits_{h=1}^H(\hat T_h-T_h)^2+\sum\limits_{h=1}^H\sum_{h \prime \neq h}(\hat T_h-T_h)(\hat T_{h \prime}-T_{h \prime})\right]\\
&\stackrel{\text{ind}}{=}\sum\limits_{h=1}^H E_\xi\left[(\hat T_h-T_h)^2\right]+\sum\limits_{h=1}^H\sum_{h \prime \neq h}E_\xi\left[(\hat T_h-T_h)\right]E_\xi\left[(\hat T_{h \prime}-T_{h \prime})\right]\\
&=\sum\limits_{h=1}^H E_\xi\left[(\hat T_h-T_h)^2\right]\\
&=\sum\limits_{h=1}^H N_h^2\frac{\sigma_h^2}{n_h}(1-f)
\end{align*}}
\end{frame}

%21
\section{Estimador de Razon}
 \tableofcontents[currentsection]

%22
\begin{frame}[allowframebreaks*]
\frametitle{Modelo para el Estimador de Razon}
\footnotesize{
Un modelo para el estimador de razon es:
$$\xi: y_k=\beta x_k+U_k \hspace{1.5cm}k=1,\dots,N$$
donde se tiene que:
\begin{align*}
&x_1,\ldots,x_N>0 \text{ y son conocidos.}\\
&y_1,\ldots,y_N \text{ son independientes.}
\end{align*}
Se tiene:
\begin{align*}
E_\xi[U_k]&=0\\
V_{\xi}[U_k]&=\sigma^2x_k\\
\end{align*}}
\end{frame}

%23
\begin{frame}[allowframebreaks*]
\frametitle{Minimos Cuadrados Ponderados}
{\footnotesize
\begin{align*}
E(Y_k)&=E(\beta x_k+U_k)=E(\beta x_k)+E(U_k)=\beta x_k\\
&\Rightarrow \hat Y_k=\hat\beta x_k\\
\end{align*}
Por minimos cuadrados ponderados se tiene:
\begin{align*}
\sum\limits_{k \in S}\left(\frac{y_k}{\sqrt{\sigma^2x_k}}-\frac{E(y_k)}{\sqrt{\sigma^2x_k}}\right)^2&=\sum\limits_{k \in S}\left(\frac{y_k}{\sqrt{\sigma^2x_k}} -\frac{\beta x_k}{\sqrt{\sigma^2x_k}}\right)^2\\
\frac{\partial SCEP}{\partial \beta}&=2\sum\limits_{k \in S}\left(\frac{y_k}{\sqrt{\sigma^2x_k}}-\frac{\hat\beta x_k}{\sqrt{\sigma^2x_k}}\right)\frac{x_k}{\sqrt{\sigma^2x_k}}=0\\
&\Rightarrow \sum\limits_{k \in S}(y_k-\hat\beta x_k)\frac{x_k}{\sigma^2x_k}=0\\
&\Rightarrow \sum\limits_{k \in S} y_k=\sum\limits_{k \in S}\hat\beta x_k\\
&\Rightarrow \hat\beta=\frac{\sum\limits_{k \in S} y_k}{\sum\limits_{k \in S} x_k}=\frac{n}{n}\frac{\sum\limits_{k \in S} y_k}{\sum\limits_{k \in S} x_k}=\frac{\bar y}{\bar x}\\
\end{align*}}
\end{frame}

%24
\begin{frame}[allowframebreaks*]
\frametitle{Insesgamiento}
{\footnotesize
Entonces $\hat T_y$ es:
\begin{align*}
\hat T_y &=\sum\limits_{k\in S}Y_k+\sum\limits_{k\in S^c}\hat Y_k\\
&=\sum\limits_{k\in S}Y_i+\hat\beta\sum\limits_{k\in S^c}x_k\\
&=\frac{\bar y}{\bar x}t_x
\end{align*}
Se va a observar si el estimador es insesgado:
\begin{align*}
E_{\xi}[\hat T_y-T]&=E_{\xi}\left[\sum\limits_{k\in S}Y_k+\hat\beta\sum\limits_{k\in S^c}x_k-\sum\limits_{k\in S}Y_k-\sum\limits_{k\in S^c}Y_k\right]\\
&=E_{\xi}\left[\hat\beta\sum\limits_{k\in S^c}x_k-\sum\limits_{k\in S^c}Y_k\right]\\
&=E_{\xi}\left[\hat\beta\sum\limits_{k\in S^c}x_k\right]-E_{\xi}\left[\sum\limits_{k\in S^c}Y_k\right]\\
&\stackrel{\text{ind}}{=}\beta\sum\limits_{k\in S^c}x_k-\beta\sum\limits_{k\in S^c}x_k\\
&=0
\end{align*}
}
\end{frame}

%25
\begin{frame}[allowframebreaks*]
\frametitle{Error Cuadratico Medio}
\begin{align*}
ECM_{\xi}[\hat T]&=V_{\xi}\left[\sum\limits_{k\in S}Y_k+\hat\beta\sum\limits_{k\in S^c}x_k-\sum\limits_{k\in S}Y_k-\sum\limits_{k\in S^c}Y_k\right]\\
&=V_{\xi}\left[\hat\beta\sum\limits_{k\in S^c}x_k-\sum\limits_{k\in S^c}Y_k\right]\\
\end{align*}
Al observarse $\hat \beta$ y $\sum\limits_{k\in S^c}Y_k$ se puede decir que son independientes dado los supuestos del modelo ya que este no depende de la muestra, por esto se puede considerar s fijo (Lohr, 2000):
\begin{equation}
=V_{\xi}\left[\hat\beta\sum\limits_{k\in S^c}x_k\right]+V_{\xi}\left[\sum\limits_{k\in S^c}Y_k\right]
\end{equation}
\end{frame}

\begin{frame}[allowframebreaks*]
\frametitle{Error Cuadratico Medio}
Observando cada uno de los terminos por separado se tiene:
\begin{align*}
V_{\xi}\left[\sum\limits_{k\in S^c}Y_k\right]&=V_{\xi}\left[\sum\limits_{k\in S^c}(\beta x_k+U_k)\right]\\
&\stackrel{ \text{ind}}{=}V_{\xi}\left[\sum\limits_{k\in S^c}\beta x_k\right]+V_{\xi}\left[\sum\limits_{k\in S^c}U_k\right]\\
&\stackrel{ \text{ind}}{=}\sum\limits_{k\in S^c}V_{\xi}\left[\beta x_k\right]+\sum\limits_{k\in S^c}V_{\xi}\left[U_k\right]\\
&=0+\sum\limits_{k\in S^c}\sigma^2x_k\\
&=\sigma^2\sum\limits_{k\in S^c}x_k
\end{align*}
\end{frame}

\begin{frame}[allowframebreaks*]
\frametitle{Error Cuadratico Medio}
Ahora
\begin{align*}
V_{\xi}\left[\hat\beta\sum\limits_{k\in S^c}x_k\right]&=\left(\sum\limits_{k\in S^c}x_k\right)^2V_{\xi}\left[\hat\beta\right]=\left(\sum\limits_{k\in S^c}x_k\right)^2V_{\xi}\left[\frac{\bar Y}{\bar X}\right]\\
&=\left(\sum\limits_{k\in S^c}x_k\right)^2V_{\xi}\left[\frac{\sum\limits_{k\in S}Y_k}{\sum\limits_{i\in S}x_k}\right]\\
&=\left(\sum\limits_{k\in S^c}x_k\right)^2\frac{1}{\left(\sum\limits_{k\in S}x_k\right)^2}V_{\xi}\left[\sum\limits_{k\in S}Y_k\right]\\
&\stackrel{\text{ind}}{=}\left(\sum\limits_{k\in S^c}x_k\right)^2\frac{1}{\sum\limits_{k\in S}x_k}\sigma^2
\end{align*}
\end{frame}

\begin{frame}[allowframebreaks*]
\frametitle{Error Cuadratico Medio}
Sumando ambos terminos
\begin{align*}
&=\left(\sum\limits_{k\in S^c}x_k\right)^2\frac{1}{\sum\limits_{k\in S}x_k}\sigma^2+\sigma^2\sum\limits_{k\in S^c}x_k\\
&=\frac{\sigma^2\sum\limits_{k\in S^c}x_k}{\sum\limits_{k\in S}x_k}\left(\sum\limits_{k\in S^c}x_k+\sum\limits_{k\in S}x_k\right)\\
&=\frac{\sigma^2}{\sum\limits_{k\in S}x_k}\sum\limits_{k\in S^c}x_k t_x=\frac{\sigma^2\left(t_x-\sum\limits_{k\in S}x_k\right)}{\sum\limits_{k\in S}x_k} t_x\\
&=\sigma^2t_x\left(\frac{t_x}{\sum\limits_{k\in S}x_k}-1\right)=\frac{\sigma^2t_x^2}{\sum\limits_{k\in S}x_k}\left(1-\frac{\sum\limits_{k\in S}x_k}{t_x}\right)
\end{align*}
\end{frame}

\begin{frame}[allowframebreaks*]
\frametitle{Error Cuadratico Medio}
Se puede observar que cuando el tama\~no de la muestra es peque\~no en comparacion del tama\~no de la poblacion entonces:
$$V_{\xi}\left[\hat T_y-T\right]\approx\frac{\sigma^2t_x^2}{\sum\limits_{k\in S}x_k}$$
Ademas:
$$\left(1-\frac{\sum\limits_{k\in S}x_k}{t_x}\right)$$
Sirve como una correccion para poblaciones finitas
\end{frame}

\end{document}
