\documentclass{article}
\usepackage[utf8]{inputenc}
\usepackage{rotating} %Rotar graficas
\usepackage{graphicx}  %Graficos
\usepackage{wrapfig} %Figura al lado del texto
\usepackage{longtable}

\title{Latex-Unidad-Uno}
\author{svergarac }
\date{April 2019}

\begin{document}

\maketitle

\section{Introduction}
\subsection{Matrices y otras expresiones matematicas}


La matriz $3 \times 3$

\[ X = \left[\begin{array}{ccc}
1 & 2 & 3 \\
4 & 5 & 6 \\
7 & 8 & 9 \end{array} \right] \]


Para representar el determinante de una matriz,  se puede usar la secuencia:
\[ det(X) = \left|\begin{array}{ccc}
6

1 & 2 & 3 \\
4 & 5 & 6 \\
7 & 8 & 9 \end{array} \right| \].

Un ejemplo de una matriz n-dimensional.Usaremos los comandos ddots
$$\ddots$, cdots $$\cdots$ e vdots$$\vdots$

\[ Y = \left[\begin{array}{cccc}
1 & 2 & \cdots & 7 \\
4 & 5 & \cdots & 6 \\
\vdots & \vdots & \ddots & \vdots \\
7 & 8 & \cdots & 4 \end{array} \right] \]

Para hacer una multiplicación de matrices,
\[ \left[\begin{array}{ccc}
21 & 12 & 6 \\
9 & 6 & 3 \\
6 & 6 & 3 \end{array} \right]=
\left[\begin{array}{cc}
1 & 5 \\
1 & 2 \\
2 & 1 \end{array} \right]
\left[\begin{array}{ccc}
1 & 2 & 1 \\
4 & 2 & 1\end{array} \right] \]

En el ambiente matricial puede ser usado para formulas, así,
\[|x| = \left\{ \begin{array}{ll}
1 & \mbox{ se } x \geq 0; \\
-1 & \mbox{ se } x < 0. \end{array} \right. \].

En el ambiente array, tambiem se pueden escribir combinatorias, como, \[ P(Y = y) = \left( \begin{array}{c}
n \\
y \end{array} \right)p^y (1-p)^{n-y} \]

Una forma alternativa mas simple es usando el comando choose, asi, \choose

\[ P(Y = y) = {n \choose y} p^y (1-p)^{n-y} \]


\subsection{Tablas}
\begin{tabular}{lrcr}
Izquierda & Derecha & Centrado & Derecha \\
Uno & Dos & Tres & Cuatro \\
1 & 2 & 3 & 4 \\
i & ii & iii & iv
\end{tabular}

\vspace{1cm}
\begin{tabular}{|l|r|c|r|}
\hline
Izquierda & Derecha & Centrado & Derecha \\
Uno & Dos & Tres & Cuatro \\ \hline
1 & 2 & 3 & 4 \\ \hline
i & ii & iii & iv \\ \hline
\end{tabular}

\vspace{1cm}

\def\tablename{Tabla}%
\begin{table}[!htb]
\caption{ Nuevos testes } \vspace*{0.2cm}
\centering
\begin{tabular}{l|r|c|r} \hline
Izquierda & Derecha & Centrado & Derecha \\ \hline \hline
Uno & Dos & Tres & Quatro \\
1 & 2 & 3 & 4 \\
i & ii & iii & iv \\
Uno & Dos & Tres & Cuatro \\
1 & 2 & 3 & 4 \\
i & ii & iii & iv \\
Un & Dos & Trs & Cuatro \\
1 & 2 & 3 & 4 \\
i & ii & iii & iv \\ \hline
\end{tabular}
\label{Tabe}
\end{table}

En la tabla \ref{Tabe}, se visualizan la alineación a la izquierda, derecha y centrado.

\def\tablename{Tabela}%
\begin{table}[ht!]
\centering
\caption{Selección de posgrados} \vspace*{0.3cm}
\begin{tabular}{c|l|l|c|c} \hline
Nivel & Año & Curso & Inscritos & Selecionados \\ \hline
Maestria & 2000 & Estadística & 15 & 10 \\ \cline{3-5}
& & Genética & 20 & 10 \\ \cline{3-5}
& & Economía & 25 & 15 \\ \cline{2-5}
& 2001 & Estadística & 18 & 10 \\ \cline{3-5}
& & Genética & 15 & 10 \\ \cline{3-5}
& & Economía & 19 & 15 \\ \hline
Doctorado & 2000 & Estadística & 10 & 5 \\ \cline{3-5}
& & Genética & 10 & 4 \\ \cline{3-5}
& & Economía & 15 & 8 \\ \cline{2-5}
& 2001 & Estadística & 8 & 4 \\ \cline{3-5}
& & Genética & 12 & 10 \\ \cline{3-5}
& & Economía & 15 & 10 \\ \hline
\end{tabular}
\end{table}

\def\tablename{Tabela}%
\begin{table}[!htb]
\centering
\caption{Notas de los alumnos del curso de \LaTeX } \vspace*{0.3cm}
\large
\begin{tabular}{l|c|rrrr} \hline
Nombre & No. TIUN  & \multicolumn{4}{c}{Calificaciones} \\ \hline
Martha & 1111112 & 5.0 & 4.8.0 & 3.8 & 4.1 \\
José & 2123333   & 3.4 & 4.5 & 3.1 & 5.0 \\
Andrés & 22235768 & 3.5 & 3.1 & \multicolumn{2}{c}{F} \\
María & 3678767 & 4.9 & 3.2& 4.1 & 3.9 \\ \hline
\end{tabular}
\end{table}


\begin{center}
\def\tablename{Tabla}%
\begin{table}[!htb]
\centering
\caption{Notas de los alumnos del curso de \LaTeX } \vspace*{0.3cm}
\begin{sideways}
\begin{tabular}{l|c|rrrr} \hline
Nombre & No. TIUN  & \multicolumn{4}{c}{Calificaciones} \\ \hline
Martha & 1111112 & 5.0 & 4.8.0 & 3.8 & 4.1 \\
José & 2123333   & 3.4 & 4.5 & 3.1 & 5.0 \\
Andrés & 22235768 & 3.5 & 3.1 & \multicolumn{2}{c}{F} \\
María & 3678767 & 4.9 & 3.2& 4.1 & 3.9 \\ \hline
\end{tabular}
\end{sideways}
\end{table}
\end{center}

\subsection{Figuras}

\begin{figure}[H]
\centering
\includegraphics[width=0.1\textwidth]{teste.jpg}
\includegraphics[width=0.2\textwidth]{teste.jpg}
\includegraphics[width=0.4\textwidth]{teste.jpg}
\caption{Escalas=0.1, 0.2 e 0.4, respectivamente}
\end{figure}

\begin{figure}[!htb]
\begin{minipage}[b]{0.40\linewidth}
\includegraphics[width=\linewidth]{teste.jpg}
\caption{Figura de la izquierda}
\label{fig1}
\end{minipage} \hfill
\begin{minipage}[b]{0.40\linewidth}
\includegraphics[width=\linewidth]{teste.jpg}
\caption{Figura a la derecha}
\label{fig2}
\end{minipage}
\end{figure}


\begin{figure}[!htb]
\begin{minipage}[b]{0.40\linewidth}
\includegraphics[width=\textwidth]{caballo.jpg}
\caption{Figura sin reflejo}
\label{fig4}
\end{minipage} \hfill
\begin{minipage}[b]{0.40\linewidth}
\reflectbox{\includegraphics[width=\textwidth]{caballo.jpg}}
\caption{Figura con reflejo}
\label{fig5}
\end{minipage}
\end{figure}

\begin{figure}[!htb]
\includegraphics[scale=0.3,angle=45]{caballo.jpg}
\caption{Figura rotación 45 grados}
\label{Aviao}
\end{figure}



\end{document}
